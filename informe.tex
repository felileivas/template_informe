%Preambulo
\documentclass[letterpaper]{article}
\usepackage[utf8x]{inputenc}
\usepackage[spanish]{babel}
\usepackage{amssymb,amsmath,amsthm,amsfonts}
\usepackage{calc}
\usepackage{graphicx}
\usepackage{subfigure}
\usepackage{gensymb}
\usepackage{natbib}
\usepackage{url}
\usepackage[utf8x]{inputenc}
\usepackage{amsmath}
\usepackage{graphicx}
\graphicspath{{images/}}
\usepackage{parskip}
\usepackage{fancyhdr}
\usepackage{vmargin}
\setmarginsrb{3 cm}{2.5 cm}{3 cm}{2.5 cm}{1 cm}{1.5 cm}{1 cm}{1.5 cm}
%%%%%%%%%%%%%%%%%%%%%%%%%%%%%%%%%%%%%%%%%%%%%%%%%%%%%%%%%%%%%%%%%%%%%%%%%%%%%%%%
%Portada
\title{Informe de Avance V 1.0}					
\author{Sebastián de Pablo
- Nicolás Diaz
- Israel Osorio
- Felipe Leiva
}
\date{\today}


\makeatletter
\let\thetitle\@title
\let\theauthor\@author
\let\thedate\@date
\makeatother

\pagestyle{fancy}
\fancyhf{}
\rhead{\theauthor}
\lhead{\thetitle}
\cfoot{\thepage}

\begin{document}

\begin{titlepage}
	\centering
    \vspace*{0.0 cm}

%Imagen o logo de portada
\includegraphics[scale = 0.13]{Logo_DuocUC.png}\\[1.0 cm]	
\textsc{\LARGE Instituto profesional DUOC UC}\\[2.0 cm]	
	\textsc{\Large DEY3101-002D}\\[0.5 cm]				
	\textsc{\large Desarrollo de software escritorio y gestión}\\[0.5 cm]	
	\rule{\linewidth}{0.2 mm} \\[0.4 cm]
	{ \huge \bfseries \thetitle}\\
	\rule{\linewidth}{0.2 mm} \\[1.5 cm]
	
	\begin{minipage}{0.4\textwidth}
		\begin{center} \large
			\emph{Autor:}\\
			\theauthor\linebreak
			\end{center}
	\end{minipage}\\[2 cm]
	
	{\large \thedate}\\[2 cm]
 
	\vfill
	
\end{titlepage}

%%%%%%%%%%%%%%%%%%%%%%%%%%%%%%%%%%%%%%%%%%%%%%%%%%%%%%%%%%%%%%%%%%%%%%%%%%%%%%%%
%Indice

\tableofcontents
\pagebreak

%%%%%%%%%%%%%%%%%%%%%%%%%%%%%%%%%%%%%%%%%%%%%%%%%%%%%%%%%%%%%%%%%%%%%%%%%%%%%%%%
%Contenido del informe
\section{Introducción}
Por encargo de la empresa Seguros BeLife se necesita desarrollar un software que manipule los datos de contratos de seguro y clientes por medio de una interfaz acorde a los pedidos del cliente. Con esto la empresa buscar actualizar su plataforma para entregar un mejor servicios beneficiándose del programa.
\section{Introducción}
Por encargo de la empresa Seguros BeLife se necesita desarrollar un software que manipule los datos de contratos de seguro y clientes por medio de una interfaz acorde a los pedidos del cliente. Con esto la empresa buscar actualizar su plataforma para entregar un mejor servicios beneficiándose del programa.
\section{Contexto}
Se necesita una plataforma acorde a los estándares actuales para operar la gestión de contratos de seguros mediante 2 módulos.

\section{Requerimientos}
Se debe permitir registro y actualización de los clientes y contratos de seguros por medio del software.
El software tiene que tener una estética acorde a la visual de la compañía.
Además el cliente entrega un manifiesto con reglas de negocios.

\section{Implementación}
El programa será implementado para ser un software para computadora en un comienzo.

Se empleara SQL Server para manipular la Base de datos.

Se utilizara el lenguaje de programación C Sharp para crear el programa.

Se validaran los campos en el programa  acorde a las reglas de negocio.

El proyecto estará bajo el paradigma de la programación orientada a objeto.


\section{Mockups}
\includegraphics[scale = 0.7]{1-inicio.png}\\[1.0 cm]
\includegraphics[scale = 0.7]{2-addcontrato.png}\\[1.0 cm]
\includegraphics[scale = 0.7]{3-addcliente.png}\\[1.0 cm]
\includegraphics[scale = 0.7]{4-cliente.png}\\[1.0 cm]
\includegraphics[scale = 0.7]{5-contrato.png}\\[1.0 cm]
\includegraphics[scale = 0.7]{6-listadocliente.png}\\[1.0 cm]
\includegraphics[scale = 0.7]{7-listadocontrato.png}\\[1.0 cm]
\end{document}